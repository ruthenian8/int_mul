% Options for packages loaded elsewhere
\PassOptionsToPackage{unicode}{hyperref}
\PassOptionsToPackage{hyphens}{url}
%
\documentclass[
]{article}
\usepackage{lmodern}
\usepackage{amssymb,amsmath}
\usepackage{ifxetex,ifluatex}
\ifnum 0\ifxetex 1\fi\ifluatex 1\fi=0 % if pdftex
  \usepackage[T1]{fontenc}
  \usepackage[utf8]{inputenc}
  \usepackage{textcomp} % provide euro and other symbols
\else % if luatex or xetex
  \usepackage{unicode-math}
  \defaultfontfeatures{Scale=MatchLowercase}
  \defaultfontfeatures[\rmfamily]{Ligatures=TeX,Scale=1}
\fi
% Use upquote if available, for straight quotes in verbatim environments
\IfFileExists{upquote.sty}{\usepackage{upquote}}{}
\IfFileExists{microtype.sty}{% use microtype if available
  \usepackage[]{microtype}
  \UseMicrotypeSet[protrusion]{basicmath} % disable protrusion for tt fonts
}{}
\makeatletter
\@ifundefined{KOMAClassName}{% if non-KOMA class
  \IfFileExists{parskip.sty}{%
    \usepackage{parskip}
  }{% else
    \setlength{\parindent}{0pt}
    \setlength{\parskip}{6pt plus 2pt minus 1pt}}
}{% if KOMA class
  \KOMAoptions{parskip=half}}
\makeatother
\usepackage{xcolor}
\IfFileExists{xurl.sty}{\usepackage{xurl}}{} % add URL line breaks if available
\IfFileExists{bookmark.sty}{\usepackage{bookmark}}{\usepackage{hyperref}}
\hypersetup{
  pdftitle={MUL in field interviews},
  pdfauthor={Daniil Ignatiev},
  hidelinks,
  pdfcreator={LaTeX via pandoc}}
\urlstyle{same} % disable monospaced font for URLs
\usepackage[margin=1in]{geometry}
\usepackage{color}
\usepackage{fancyvrb}
\newcommand{\VerbBar}{|}
\newcommand{\VERB}{\Verb[commandchars=\\\{\}]}
\DefineVerbatimEnvironment{Highlighting}{Verbatim}{commandchars=\\\{\}}
% Add ',fontsize=\small' for more characters per line
\usepackage{framed}
\definecolor{shadecolor}{RGB}{248,248,248}
\newenvironment{Shaded}{\begin{snugshade}}{\end{snugshade}}
\newcommand{\AlertTok}[1]{\textcolor[rgb]{0.94,0.16,0.16}{#1}}
\newcommand{\AnnotationTok}[1]{\textcolor[rgb]{0.56,0.35,0.01}{\textbf{\textit{#1}}}}
\newcommand{\AttributeTok}[1]{\textcolor[rgb]{0.77,0.63,0.00}{#1}}
\newcommand{\BaseNTok}[1]{\textcolor[rgb]{0.00,0.00,0.81}{#1}}
\newcommand{\BuiltInTok}[1]{#1}
\newcommand{\CharTok}[1]{\textcolor[rgb]{0.31,0.60,0.02}{#1}}
\newcommand{\CommentTok}[1]{\textcolor[rgb]{0.56,0.35,0.01}{\textit{#1}}}
\newcommand{\CommentVarTok}[1]{\textcolor[rgb]{0.56,0.35,0.01}{\textbf{\textit{#1}}}}
\newcommand{\ConstantTok}[1]{\textcolor[rgb]{0.00,0.00,0.00}{#1}}
\newcommand{\ControlFlowTok}[1]{\textcolor[rgb]{0.13,0.29,0.53}{\textbf{#1}}}
\newcommand{\DataTypeTok}[1]{\textcolor[rgb]{0.13,0.29,0.53}{#1}}
\newcommand{\DecValTok}[1]{\textcolor[rgb]{0.00,0.00,0.81}{#1}}
\newcommand{\DocumentationTok}[1]{\textcolor[rgb]{0.56,0.35,0.01}{\textbf{\textit{#1}}}}
\newcommand{\ErrorTok}[1]{\textcolor[rgb]{0.64,0.00,0.00}{\textbf{#1}}}
\newcommand{\ExtensionTok}[1]{#1}
\newcommand{\FloatTok}[1]{\textcolor[rgb]{0.00,0.00,0.81}{#1}}
\newcommand{\FunctionTok}[1]{\textcolor[rgb]{0.00,0.00,0.00}{#1}}
\newcommand{\ImportTok}[1]{#1}
\newcommand{\InformationTok}[1]{\textcolor[rgb]{0.56,0.35,0.01}{\textbf{\textit{#1}}}}
\newcommand{\KeywordTok}[1]{\textcolor[rgb]{0.13,0.29,0.53}{\textbf{#1}}}
\newcommand{\NormalTok}[1]{#1}
\newcommand{\OperatorTok}[1]{\textcolor[rgb]{0.81,0.36,0.00}{\textbf{#1}}}
\newcommand{\OtherTok}[1]{\textcolor[rgb]{0.56,0.35,0.01}{#1}}
\newcommand{\PreprocessorTok}[1]{\textcolor[rgb]{0.56,0.35,0.01}{\textit{#1}}}
\newcommand{\RegionMarkerTok}[1]{#1}
\newcommand{\SpecialCharTok}[1]{\textcolor[rgb]{0.00,0.00,0.00}{#1}}
\newcommand{\SpecialStringTok}[1]{\textcolor[rgb]{0.31,0.60,0.02}{#1}}
\newcommand{\StringTok}[1]{\textcolor[rgb]{0.31,0.60,0.02}{#1}}
\newcommand{\VariableTok}[1]{\textcolor[rgb]{0.00,0.00,0.00}{#1}}
\newcommand{\VerbatimStringTok}[1]{\textcolor[rgb]{0.31,0.60,0.02}{#1}}
\newcommand{\WarningTok}[1]{\textcolor[rgb]{0.56,0.35,0.01}{\textbf{\textit{#1}}}}
\usepackage{graphicx}
\makeatletter
\def\maxwidth{\ifdim\Gin@nat@width>\linewidth\linewidth\else\Gin@nat@width\fi}
\def\maxheight{\ifdim\Gin@nat@height>\textheight\textheight\else\Gin@nat@height\fi}
\makeatother
% Scale images if necessary, so that they will not overflow the page
% margins by default, and it is still possible to overwrite the defaults
% using explicit options in \includegraphics[width, height, ...]{}
\setkeys{Gin}{width=\maxwidth,height=\maxheight,keepaspectratio}
% Set default figure placement to htbp
\makeatletter
\def\fps@figure{htbp}
\makeatother
\setlength{\emergencystretch}{3em} % prevent overfull lines
\providecommand{\tightlist}{%
  \setlength{\itemsep}{0pt}\setlength{\parskip}{0pt}}
\setcounter{secnumdepth}{-\maxdimen} % remove section numbering
\ifluatex
  \usepackage{selnolig}  % disable illegal ligatures
\fi

\title{MUL in field interviews}
\author{Daniil Ignatiev}
\date{4 РёСЋРЅСЏ 2021 Рі}

\begin{document}
\maketitle

\hypertarget{mul-msl-in-field-interviews}{%
\subsection{MUL \& MSL in field
interviews}\label{mul-msl-in-field-interviews}}

\hypertarget{abstract}{%
\subsubsection{Abstract}\label{abstract}}

The paper explores several linguistic questions, making use of the RSUH
folklore archive. The first of them is how traditional sociolinguistic
variables, namely the mean utterance length and the mean sentence,
length relate to the gender of the speaker and to the gender of the
addressee. The second question is how well we can study the former
matter, given the current structure of the corpus and what the corpus
currently lacks in this respect. Thirdly, we provide some conclusions on
how collectors' genders could influence the quantity of the collected
material.

\hypertarget{introduction}{%
\subsubsection{Introduction}\label{introduction}}

Mean utterance length (MLU) and mean sentence length (MSU) are both
traditional variables that have been excessively studied in relation to
sociolinguistic factors. What makes a study of such kind especially
interesting is the stereotype than women generally talk more than men
do. This motivated a lot of researchers to address the matter on the
material of different languages (see Daniel \& Zelenkov 2012, Tannen
1990). Despite the fact, that the problem is generally well-studied, we
still raise it thanks to the specific properties of our corpus that
offer a new perspective on the old research object. If we compare our
study to the one M. A. Daniel performed on the data from the Russian
National Corpus, we can see one important advantage of our dataset,
namely the pragmatic uniformity. The data available in the Oral
Subcorpus of the RNC comes from very different sources, including purely
artificial ones, like films or plays, and therefore this data unifies
very different examples of speech in terms of pragmatics, which a
researcher can hardly account for. Our dataset on the other hand
consists of transcribed field recordings and interviews that more or
less belong to a same type of communicative situations. This fact makes
the statistical hypotheses we intend to test a lot more convincing.

\hypertarget{preparation}{%
\subsubsection{Preparation}\label{preparation}}

The dataset was previously extracted from the RSUH archive and prepared
using Python scripts. The original archive consists of some 24000
entries, each of which contains one or several answers to interviewers'
questions by different speakers. All the questions or interviewers'
remarks are enclosed in square brackets, which makes it easy to filter
them by using regular expressions. When calculating the mean utterance
length, we considered the span between two questions or remarks a single
utterance and thus the mean number of tokens inside those spans was
viewed as the MLU for each entry. The mean sentence length on the other
hand was calculated after all the interviewers' utterances were
excluded.

\begin{Shaded}
\begin{Highlighting}[]
\NormalTok{dataset }\OtherTok{\textless{}{-}} \FunctionTok{read.csv}\NormalTok{(}\StringTok{"https://raw.githubusercontent.com/ruthenian8/int\_mul/master/uq\_infs\_genders.csv"}\NormalTok{)}
\end{Highlighting}
\end{Shaded}

\includegraphics{int_mul_files/figure-latex/unnamed-chunk-3-1.pdf}
\includegraphics{int_mul_files/figure-latex/unnamed-chunk-4-1.pdf}

In the histograms above the red line shows the median value. Filtering
the dataset, we exclude entries in which the mean sentence length
exceeds 40 or the mean utterance length exceeds 120, as those can be
viewed as outliers that might affect the comparison results. This step
will also make both distributions more or less symmetrical around the
median, which in its turn will simplify the upcoming tests.

\begin{Shaded}
\begin{Highlighting}[]
\NormalTok{dataset }\OtherTok{\textless{}{-}}\NormalTok{ dataset }\SpecialCharTok{\%\textgreater{}\%}
  \FunctionTok{filter}\NormalTok{(inf\_gender1 }\SpecialCharTok{==} \StringTok{"f"} \SpecialCharTok{|}\NormalTok{ inf\_gender1 }\SpecialCharTok{==} \StringTok{"m"}\NormalTok{) }\SpecialCharTok{\%\textgreater{}\%} 
  \FunctionTok{filter}\NormalTok{(mean\_sl }\SpecialCharTok{\textless{}=} \DecValTok{40} \SpecialCharTok{\&}\NormalTok{ mean\_ul }\SpecialCharTok{\textless{}} \DecValTok{120}\NormalTok{)}
\end{Highlighting}
\end{Shaded}

The three factors that generally need to be taken care of when comparing
parameters like the MUL are independence of observations, normal
distribution of data and homosedasticity (equivalence of variance). The
first of this constraints was taken care of, when we selected our
examples from the archive, as we included only one randomly selected
entry from each of the speakers. This means that none of the examples in
the dataset was influenced by another. The two other factors are
commonly viewed as less restrictive for several reasons.

\begin{itemize}
\tightlist
\item
  Firstly, since the impact that a non-normal distribution of data would
  make on a t-test can easily be avoided by resorting to Wilcoxon's test
  instead. This possibility greatly alleviates the construction of our
  dataset, since the graph clearly suggests that the distribution of
  data is quite skewed and even resembles Poisson's distribution (see
  figures below).
\item
  Secondly, while we still intend to check the homosedasticity of the
  data, using the Wilcoxon's test or the version of the t-test, known as
  Welch's independent sample t-test reduces the influence of this
  factor.\\
  These two assumptions lead us to the conclusion that the current state
  of the dataset is acceptable for comparing mean values inside any
  groups of choice.
\end{itemize}

Another assumption that should be accounted for is that the
distributions should be symmetrical around the median for successfully
running the Wilcoxon's test. This was almost the case before filtering
out the outliers, as the graphs suggest. We can hope that after the
cleanup has been performed, the data finally meets this requirement.

The relevant variables that are present in the dataset are:

\begin{itemize}
\tightlist
\item
  text: includes a full version of the text for each of the entries.
\item
  mean\_ul: mean utterance length, calculated in the fashion described
  above
\item
  mean\_sl: mean sentence length,
\item
  inf\_gender1: the gender of the speaker
\item
  sob\_gender1: the gender of the first interviewer
\item
  sob\_gender2: the gender of the second interviewer (if present, "" if
  absent)
\item
  sob\_gender3 \&
\item
  sob\_gender4: genders of the third and the fourth interviewer
\end{itemize}

The four latter parameters allow us to separate the entries in several
groups depending on the gender of the speakers. Thus, we can compare the
cases, in which the speaker has a conversation either with a single
interviewer of a certain gender or with a team of interviewers, that can
be either diverse or uniform in terms of gender. This aspect of the
situation can possibly influence the speaker, determining, whether they
wish to share their knowledge. For instance, it was noted by the
specialists that informers are much more eager to share their knowledge
about magic and other forbidden subjects with the interviewers that they
perceive as equal to themselves.

\hypertarget{experiments}{%
\subsubsection{Experiments}\label{experiments}}

After the requirements have been accounted for, we can proceed to the
comparison of groups. First of all, we are going to compare speaker
genders overall, without making any further distinctions.

\includegraphics{int_mul_files/figure-latex/unnamed-chunk-6-1.pdf}

In both cases the third quartile boundary is higher for the female
group, although we may assume that this tendency is due to the unequal
sample sizes, as the female group includes more entries. The white
points that mark the mean of the distribution suggest that little
difference is present, but we are still going to run the corresponding
tests. We also calculate the corresponding r-statistic so as to
determine the effect size.

\hypertarget{mul}{%
\paragraph{MUL}\label{mul}}

\begin{verbatim}
## # A tibble: 1 x 8
##   .y.     group1 group2    n1    n2 statistic      p p.signif
## * <chr>   <chr>  <chr>  <int> <int>     <dbl>  <dbl> <chr>   
## 1 mean_ul f      m        718   228    88968. 0.0477 *
\end{verbatim}

\begin{verbatim}
## # A tibble: 1 x 7
##   .y.     group1 group2 effsize    n1    n2 magnitude
## * <chr>   <chr>  <chr>    <dbl> <int> <int> <ord>    
## 1 mean_ul f      m       0.0644   718   228 small
\end{verbatim}

\hypertarget{msl}{%
\paragraph{MSL}\label{msl}}

\begin{verbatim}
## # A tibble: 1 x 8
##   .y.     group1 group2    n1    n2 statistic       p p.signif
## * <chr>   <chr>  <chr>  <int> <int>     <dbl>   <dbl> <chr>   
## 1 mean_sl f      m        718   228    92590. 0.00281 **
\end{verbatim}

\begin{verbatim}
## # A tibble: 1 x 7
##   .y.     group1 group2 effsize    n1    n2 magnitude
## * <chr>   <chr>  <chr>    <dbl> <int> <int> <ord>    
## 1 mean_sl f      m       0.0971   718   228 small
\end{verbatim}

Our interpretation of the tests above depends on what alpha we choose
for testing the hypotheses. As long as we pick an alpha of 0.001, like
M. A. Daniel did in his research, the difference of means in both of the
tests has no statistical significance. However, if we choose an alpha of
0.05 that is generally acknowledged as sufficient for the social
sciences, the results appear to be significant. Whichever fashion we go
for, the r-statistic still implies that the effect size is quite small
in both of the tests. This fact allows us to conclude that according to
the data that we have, the common assumption that women speak more is
not supported statistically. Having determined this conclusion, we can
now try to split the groups into smaller fractions, so as to see, if any
patterns can be spotted in this manner.

Although we stated at the beginning, that the corpus that we are working
with is generally uniform in terms of pragmatics, speakers' perception
of the communicative situation can still differ, depending on the number
of interviewers they converse with. As long as only one interviewer
participates in the dialogue, the situation can possibly be viewed as a
private talk, which does not force the informant to change his everyday
speaking patterns. Meanwhile a conversation with multiple interviewers
(normally from 2 to 4) is more likely to be percieved as a formal
situation that requires the informant to make respective adjustments to
their manner of speaking. Therefore it would be interesting to compare
the corresponding corpus entries. Firstly, we can compare the subgroups
determined by the exact number of interviewers.

\includegraphics{int_mul_files/figure-latex/unnamed-chunk-9-1.pdf}

Then we may also look at the difference between the formal
(\textgreater2 interviewers) and the informal subgroups. The sample
sizes are 815 for formal and 131 for informal types.

\begin{verbatim}
## # A tibble: 2 x 2
##   sit.type     n
## * <chr>    <int>
## 1 formal     815
## 2 informal   131
\end{verbatim}

\includegraphics{int_mul_files/figure-latex/unnamed-chunk-11-1.pdf}

What we can see from the graph is that the assumption we made above is
presumably false, as the means and the medians of the two distributions
are roughly the same. The notches also clearly intersect, which implies
the lack of statistically significant differences between the medians.
Still, it can also be noted that the whisker ends are different and
include higher values in the case of the formal situation type. The
other difference is that the plot for the formal type includes many more
outliers with high MUL values, although this fact can be explained by
the difference of sample sizes. However, what would be interesting to
test is whether gender-based distinctions exist in the two new groups.
Firstly, we will create plots for both the formal and the informal case.

\includegraphics{int_mul_files/figure-latex/unnamed-chunk-12-1.pdf}

The graph for the informal group suggests that no differences are
present. The graph for the formal group on the other hand shows that the
quartile borders and the whiskers cover a narrower range and find
themselves lower in the case of the male speaker group. The notches of
the boxes are also differently positioned, which hints at the difference
of medians. Running the same comparison tests inside the formal
subgroup, we can see, that the significance between speaker genders is
more notable in this particular case.

\hypertarget{mul-1}{%
\paragraph{MUL}\label{mul-1}}

\begin{verbatim}
## # A tibble: 1 x 8
##   .y.     group1 group2    n1    n2 statistic      p p.signif
## * <chr>   <chr>  <chr>  <int> <int>     <dbl>  <dbl> <chr>   
## 1 mean_ul f      m        618   197    67602. 0.0194 *
\end{verbatim}

\begin{verbatim}
## # A tibble: 1 x 7
##   .y.     group1 group2 effsize    n1    n2 magnitude
## * <chr>   <chr>  <chr>    <dbl> <int> <int> <ord>    
## 1 mean_ul f      m       0.0819   618   197 small
\end{verbatim}

\hypertarget{msl-1}{%
\paragraph{MSL}\label{msl-1}}

\begin{verbatim}
## # A tibble: 1 x 8
##   .y.     group1 group2    n1    n2 statistic        p p.signif
## * <chr>   <chr>  <chr>  <int> <int>     <dbl>    <dbl> <chr>   
## 1 mean_sl f      m        618   197    70514. 0.000804 ***
\end{verbatim}

\begin{verbatim}
## # A tibble: 1 x 7
##   .y.     group1 group2 effsize    n1    n2 magnitude
## * <chr>   <chr>  <chr>    <dbl> <int> <int> <ord>    
## 1 mean_sl f      m        0.117   618   197 small
\end{verbatim}

While the difference of the mean utterance lengths still does not have
to be taken too seriously, as the p-value shows a high probability of a
type-1 error, while the r-statistic value is small, the difference of
mean sentence lengths appears to be quite notable. If we try to
interpret these distinctions in terms of some real-world tendencies, we
might suppose that female speakers are more eager to adapt their
speaking patterns to the specific communicative situation. It may be
either due to differences in the perception of the situation (members of
one group view it as responsible, while others do not) or due to some
other factor. So far the tendencies that we observed were hardly
significant, but still informative. Nevertheless, the dataset can be
fractured even more, if we take into account the gender of the
interviewers. At first we can take a look at how the distributions look,
when there is only one interviewer.

\includegraphics{int_mul_files/figure-latex/unnamed-chunk-15-1.pdf}

\hypertarget{mul-2}{%
\paragraph{MUL}\label{mul-2}}

\begin{verbatim}
## # A tibble: 4 x 3
##   part.genders mean_MUL     n
## * <chr>           <dbl> <int>
## 1 F2F              18.8    55
## 2 F2M              20.9    45
## 3 M2F              21.2    14
## 4 M2M              25.6    17
\end{verbatim}

\hypertarget{msl-2}{%
\paragraph{MSL}\label{msl-2}}

\begin{verbatim}
## # A tibble: 4 x 3
##   part.genders mean_MSL     n
## * <chr>           <dbl> <int>
## 1 F2F              6.91    55
## 2 F2M              7.89    45
## 3 M2F              7.05    14
## 4 M2M              7.77    17
\end{verbatim}

Judging by the graphs, we can conclude that no real difference between
the means is present, although both the MUL and the MSL seem to be
slightly higher, when the interviewer is male. Just to be sure, we may
apply an ANOVA with TukeyHSD to check this out. The reason, why we chose
TukeyHSD is because this post-hoc test has less requirements than a
pairwise t.test.

\hypertarget{mul-3}{%
\paragraph{MUL}\label{mul-3}}

\begin{verbatim}
##               Df Sum Sq Mean Sq F value Pr(>F)
## part.genders   3    607   202.4   0.494  0.687
## Residuals    127  52029   409.7
\end{verbatim}

\begin{verbatim}
##   Tukey multiple comparisons of means
##     95% family-wise confidence level
## 
## Fit: aov(formula = mean_ul ~ part.genders, data = data_single)
## 
## $part.genders
##              diff        lwr      upr     p adj
## F2M-F2F 2.1515373  -8.440298 12.74337 0.9519426
## M2F-F2F 2.3809234 -13.392937 18.15478 0.9793308
## M2M-F2F 6.7672403  -7.855179 21.38966 0.6249252
## M2F-F2M 0.2293861 -15.896164 16.35494 0.9999817
## M2M-F2M 4.6157030 -10.385425 19.61683 0.8538180
## M2M-M2F 4.3863170 -14.631084 23.40372 0.9317390
\end{verbatim}

\hypertarget{msl-3}{%
\paragraph{MSL}\label{msl-3}}

\begin{verbatim}
##               Df Sum Sq Mean Sq F value Pr(>F)
## part.genders   3   27.9   9.311   0.612  0.608
## Residuals    127 1931.3  15.207
\end{verbatim}

\begin{verbatim}
##   Tukey multiple comparisons of means
##     95% family-wise confidence level
## 
## Fit: aov(formula = mean_sl ~ part.genders, data = data_single)
## 
## $part.genders
##               diff       lwr      upr     p adj
## F2M-F2F  0.9797111 -1.060959 3.020381 0.5964374
## M2F-F2F  0.1421809 -2.896881 3.181243 0.9993509
## M2M-F2F  0.8606398 -1.956580 3.677860 0.8564542
## M2F-F2M -0.8375301 -3.944350 2.269290 0.8961965
## M2M-F2M -0.1190713 -3.009255 2.771113 0.9995562
## M2M-M2F  0.7184589 -2.945518 4.382436 0.9564613
\end{verbatim}

As the reports suggest, the slight difference that exists is of no
statistical significance.

We may also look at the situations, in which there are two or more
interviewers. For this comparison we introduce a variable that has
values ``Fonly'' and ``Monly'' if the interviewer group is composed
solely of females or males respectively. All other composition types are
marked as ``Mixed''. After combining this value with the gender of the
speaker we are left with 6 distinct dialogue types.

\includegraphics{int_mul_files/figure-latex/unnamed-chunk-21-1.pdf}

\includegraphics{int_mul_files/figure-latex/unnamed-chunk-22-1.pdf}

Although the graphs show no drastic difference between the
distributions, the means are obviously not equal, which is why we may
compare them using the ANOVA test and utilizing the Tukey Honest
Significant differences as a post-hoc comparison.

\hypertarget{mul-4}{%
\paragraph{MUL}\label{mul-4}}

\begin{verbatim}
## # A tibble: 6 x 3
##   inf2group   MUL   num
## * <chr>     <dbl> <int>
## 1 f2Fonly    19.5   268
## 2 f2Mixed    21.3   235
## 3 f2Monly    20.3   115
## 4 m2Fonly    17.9    65
## 5 m2Mixed    17.2    62
## 6 m2Monly    18.5    70
\end{verbatim}

\begin{verbatim}
##              Df Sum Sq Mean Sq F value Pr(>F)
## inf2group     5   1344   268.7   0.732    0.6
## Residuals   809 296958   367.1
\end{verbatim}

\begin{verbatim}
##   Tukey multiple comparisons of means
##     95% family-wise confidence level
## 
## Fit: aov(formula = mean_ul ~ inf2group, data = grouped)
## 
## $inf2group
##                       diff        lwr       upr     p adj
## f2Mixed-f2Fonly  1.7661861  -3.124834  6.657206 0.9073024
## f2Monly-f2Fonly  0.7429578  -5.358026  6.843941 0.9993301
## m2Fonly-f2Fonly -1.6623699  -9.229214  5.904475 0.9889789
## m2Mixed-f2Fonly -2.3325552 -10.045327  5.380217 0.9549011
## m2Monly-f2Fonly -0.9933902  -8.339522  6.352741 0.9988865
## f2Monly-f2Mixed -1.0232283  -7.251506  5.205049 0.9971660
## m2Fonly-f2Mixed -3.4285560 -11.098405  4.241293 0.7976608
## m2Mixed-f2Mixed -4.0987414 -11.912594  3.715111 0.6654218
## m2Monly-f2Mixed -2.7595763 -10.211763  4.692611 0.8978159
## m2Fonly-f2Monly -2.4053277 -10.898060  6.087404 0.9658914
## m2Mixed-f2Monly -3.0755131 -11.698518  5.547492 0.9116867
## m2Monly-f2Monly -1.7363480 -10.033035  6.560339 0.9911808
## m2Mixed-m2Fonly -0.6701854 -10.385707  9.045336 0.9999591
## m2Monly-m2Fonly  0.6689797  -8.758117 10.096076 0.9999529
## m2Monly-m2Mixed  1.3391651  -8.205460 10.883790 0.9986695
\end{verbatim}

\hypertarget{msl-4}{%
\paragraph{MSL}\label{msl-4}}

\begin{verbatim}
## # A tibble: 6 x 3
##   inf2group   MSL   num
## * <chr>     <dbl> <int>
## 1 f2Fonly    7.83   268
## 2 f2Mixed    7.48   235
## 3 f2Monly    6.96   115
## 4 m2Fonly    6.32    65
## 5 m2Mixed    7.05    62
## 6 m2Monly    6.38    70
\end{verbatim}

\begin{verbatim}
##              Df Sum Sq Mean Sq F value Pr(>F)  
## inf2group     5    220   44.02   2.647  0.022 *
## Residuals   809  13455   16.63                 
## ---
## Signif. codes:  0 '***' 0.001 '**' 0.01 '*' 0.05 '.' 0.1 ' ' 1
\end{verbatim}

\begin{verbatim}
##   Tukey multiple comparisons of means
##     95% family-wise confidence level
## 
## Fit: aov(formula = mean_sl ~ inf2group, data = grouped)
## 
## $inf2group
##                        diff       lwr       upr     p adj
## f2Mixed-f2Fonly -0.34446198 -1.385567 0.6966430 0.9345818
## f2Monly-f2Fonly -0.86078667 -2.159445 0.4378718 0.4069028
## m2Fonly-f2Fonly -1.50655330 -3.117236 0.1041291 0.0820599
## m2Mixed-f2Fonly -0.77218640 -2.413931 0.8695582 0.7605133
## m2Monly-f2Fonly -1.44473834 -3.008440 0.1189630 0.0891772
## f2Monly-f2Mixed -0.51632469 -1.842079 0.8094296 0.8762718
## m2Fonly-f2Mixed -1.16209132 -2.794699 0.4705166 0.3242764
## m2Mixed-f2Mixed -0.42772442 -2.090985 1.2355363 0.9776147
## m2Monly-f2Mixed -1.10027636 -2.686553 0.4860000 0.3539132
## m2Fonly-f2Monly -0.64576663 -2.453534 1.1620007 0.9111472
## m2Mixed-f2Monly  0.08860026 -1.746897 1.9240975 0.9999931
## m2Monly-f2Monly -0.58395168 -2.349989 1.1820853 0.9347467
## m2Mixed-m2Fonly  0.73436689 -1.333684 2.8024177 0.9131974
## m2Monly-m2Fonly  0.06181496 -1.944842 2.0684716 0.9999993
## m2Monly-m2Mixed -0.67255194 -2.704226 1.3591218 0.9344439
\end{verbatim}

Once again, the difference seems to be to small to be considered.
Although the ANOVA shows that not all groups have a similar mean MSL,
the post-hoc test proves that the differences are not significant.
However, we can state that the difference between the female-to-female
group and two other groups, namely male-to-female and male-to-male are
on the verge of statistical significance. Of course, the latter fact
does not allow us to make any strict conclusions.

\hypertarget{discussion}{%
\subsubsection{Discussion}\label{discussion}}

\begin{itemize}
\tightlist
\item
  It may seem from the experiments above that we failed to achieve any
  significant result, as we did not find any notable tendencies.
  However, the main goal of the study was not to confirm, but to test a
  common assumption about the communicative behaviour of certain
  genders. In this sense the experiment can be called a success, as our
  study, made on a pragmatically uniform corpus, proves that the
  gender-based differences are not statistically significant.
\item
  The study also showed some disadvantages of the corpus, like the lack
  of an age variable or like the lack of balance in terms of speaker
  genders. These weak points can potentially be taken into account, when
  collecting new data.
\item
  As for the interviewer team composition in terms of gender, the study
  shows that the effect that the corresponding variables make is hardly
  noticeable - at least as far as the mean utterance length and the mean
  sentence length are concerned. We would need some markup for speaker's
  age and for the age of the interviewers to study this question in
  detail.
\end{itemize}

\end{document}
